\documentclass[11pt]{article}
\usepackage[margin=1in]{geometry}
\usepackage[pdftex]{graphicx}
\usepackage{amsmath,amssymb,amsthm}
\usepackage{custom}

% color boxes
\usepackage[many]{tcolorbox}
\tcbuselibrary{theorems}
\newtcbtheorem{psexample}{Example}{colback=green!10,colframe=green!40!black,fonttitle=\bfseries,breakable,enhanced}{ex}
\newtcbtheorem{psproblem}{Problem}{colback=blue!10,colframe=blue!40!black,fonttitle=\bfseries,breakable,enhanced}{id}
\newtcbtheorem{psremark}{Remark}{colback=purple!10,colframe=purple!40!black,fonttitle=\bfseries,breakable,enhanced}{rm}
\newtcbtheorem{pssolution}{Solution}{colback=orange!10,colframe=orange!40!black,fonttitle=\bfseries,breakable,enhanced}{sn}
\tcbset{noparskip/.style={before={\par\pagebreak[0]\medskip\parindent=0pt},
after={\par\medskip}}}

% formatting for problems and solutions
\definecolor{ptred}{rgb}{0.7,0.1,0.1}
\newcommand{\ptfmt}[1]{\textbf{\color{ptred}#1\color{black}}}
\definecolor{advblue}{rgb}{0.1,0.1,0.7}
\newcommand{\advanced}{\!\textbf{\color{advblue}[A]\color{black}}\ }

\newtheorem*{solution}{Solution}
\newtheorem*{answer}{Answer}

\makeatletter
\newtheoremstyle{mystyle}
{\topsep}               % space above
{\topsep}               % space below
{}                      % bodyfont
{}                      % indent
{\bfseries}             % headfont
{}                      % punctuation
{0.6em}                 % space after head
{\llap{\hspace{.6em}}\thmname{#1}\thmnumber{ #2}\thmnote{\normalfont{ (#3)}}{\bfseries .}}  %theoremheadspec
\theoremstyle{mystyle}
\newtheorem{pproblem}{Problem}
\makeatother

% clocks for time limits
\usepackage{clock} 
\ClockFrametrue\ClockStyle=1

% headers and footers
\usepackage{fancyhdr}
\pagestyle{fancy}
\lhead{Filip Ramazan}
\chead{}
\rhead{STAT 461}
\lfoot{}
\cfoot{\thepage}
\rfoot{}
\renewcommand{\headrulewidth}{0.4pt}
\setlength{\headheight}{14pt}

\newcommand{\psettitle}[1]{
    \begin{center}
    \huge \textbf{#1}
    \end{center}
}

\linespread{1.03} % give a little extra room
\setlength{\parindent}{0.2in} % reduce paragraph indent a bit

% uncomment to hide solutions
% \usepackage{environ}
% \NewEnviron{hide}{}
% \let\solution\hide
% \let\endsolution\endhide
% \let\answer\hide
% \let\endanswer\endhide

\begin{document}

\psettitle{Homework \#3}

\noindent

\begin{psproblem}{The Power of the Polynomial}{}
What is the coefficient of $x^3$ in the expansion of ${(2-3x)}^5$?
\end{psproblem}

\begin{solution}
By Newton's binomial formula, we know that \[(a + b)^n = \sum_{k=0}^{n} \dbinom{n}{k} a^{n-k} b^k .\]
Hence, for $(2-3x)^5$, we know that that for $x^3$, the term is as such: $$\dbinom{5}{2} \cdot {(-3x)}^{(5-2)} \cdot 2^2 =  -\dfrac{5!}{2!\cdot3!} \cdot 27 \cdot 4 \cdot x^3=-1080x^3.$$
The coffiecient, thus, is \boxed{-1080}.
\end{solution}

\begin{psproblem}{Coefficients Count}{}
What is the coefficient of $a^3b^2c^4$ in the expansion of $(a + b + c)^9$?
\end{psproblem}

\begin{solution}
We solve for a formula for the trinomial expansion:
\[(a+\underbrace{b+c}_{=d})^p = \sum_{k=0}^{p} \dbinom{p}{k} a^{p-k} d^k = 
\sum_{k=0}^{p} \dbinom{p}{k} a^{p-k} (b+c)^{k} =
\sum_{k=0}^{p} \dbinom{p}{k} a^{p-k} \left( \sum_{i=0}^{k} \dbinom{k}{i} b^{k-i} c^{i} \right).
\]
Hence, for $(a+b+c)^9$, we can calculate the coeffcient of $a^3b^2c^4$ as follows: 
\[
Ca^3b^2c^4=\dbinom{9}{6}a^{9-6} \left[ \dbinom{6}{4}b^{6-4}c^4 \right] \implies C=\dbinom{9}{6}\dbinom{6}{4}=\dfrac{9!}{6!\cdot3!} \cdot \dfrac{6!}{4!\cdot2!}=20 \cdot 63=\boxed{1260}
\]
\end{solution}

\begin{psproblem}{Digit Drama - Odds Edition}{}
An integer is selected randomly between 100 and 999. What is the probability that the number has distinct digits?
\end{psproblem}

\begin{solution}
To pick the first digit, there are 9 options (excluding 0), for the second also 9 (including 0, excluding the first picked digit), and for the third there are 8 options (one excluding the first and second digits). We use the fundamental theorem of probability to calculate the total number of ways we can pick: $9 \cdot 9 \cdot 8 = 648$. Thus, the total probability is: $\dfrac{648}{900}=\boxed{\dfrac{18}{25}}$.
\end{solution}

\begin{psproblem}{Twin Rivalry}{}
Three students are selected randomly from a group of twelve, including twins Anna and Brian. What is the probability that Anna is selected, but Brian is not?
\end{psproblem}

\begin{solution}
\[ P(\text{Anna selected but not Brian})=\dfrac{\text{favorable outcomes}}{\text{total outcomes}}=
\dfrac{\dbinom{10}{2}}{\dbinom{12}{3}}=\boxed{\dfrac{9}{44}}
\]
We know that there are $\binom{10}{2}$ favorable outcomes because we assume that Anna is already given a spot, and we choose 2 more spots out of 10 people, which excludes Brian.
\end{solution}

\begin{psproblem}{Marble Mayhem}{}
Three marbles are selected randomly from a bag containing 6 yellow marbles and 4 green marbles. What is the probability that there will be more green marbles than yellow marbles selected?
\end{psproblem}

\begin{solution}
For there to be green marbles than yellow marbles selected, we can either have 3 green and 0 yellow or 2 green and 1 yellow. We calculate the number of favorable outcomes over unfavorable:
\[
P=\dfrac{\dbinom{4}{2} \cdot \dbinom{6}{1} + \dbinom{4}{3}}{\dbinom{10}{3}}=\dfrac{6\cdot 6+4}{120}=\boxed{\dfrac{1}{3}}
\]
\end{solution}

\begin{psproblem}{Selecting Several Socks}{}
We select randomly two socks from a drawer containing 4 white socks, 6 blue socks, and 2 red socks. What is the probability that the socks will match?
\end{psproblem}

\begin{solution}
\[ P(\text{socks will match}) = \dfrac{\dbinom{4}{2}+\dbinom{6}{2}+\dbinom{2}{2}}{\dbinom{12}{2}}=\dfrac{22}{66}=\boxed{\dfrac{1}{3}}\]
\end{solution}

\begin{psproblem}{Pair-ly Lucky}{}
Five cards are dealt from a regular deck of 52 cards. What is the probability of being dealt \textit{one pair}?
\end{psproblem}

\begin{solution}
The total amount of ways five cards can be picked is $\binom{52}{5}$. There are 13 ranks, and four identical cards of each rank. Therefore, we can select two cards from each quartet, and select out of the remaining 12 ranks, 3 ranks and one of four cards from each, to avoid any extra pairs. Therefore, we have the following probability:
\[ P(\text{one pair from a deck})=\dfrac{\dbinom{4}{2} \cdot 13 \cdot \dbinom{12}{3} \cdot 64}{\dbinom{52}{5}} = \dfrac{13 \cdot 6 \cdot 220 \cdot 64}{2598960}=\dfrac{22\cdot13\cdot16}{10829}=\boxed{0.422569}\]
\end{solution}


\begin{psproblem}{One and Done}{}
\begin{enumerate}[label=\alph*.]
\item What is the probability of having at least one 1 in four rolls of a die?
\item What is the probability of having at least one double 1 in 24 rolls of two dice?
\end{enumerate}
\end{psproblem}

\begin{solution}
\leavevmode
\begin{enumerate}[label=\alph*.]
\item Having at least one 1 in four rolls is the complement of not having any, so the probability is: 
\[
P(\text{at least one 1}) = 1 - \left(\dfrac{5}{6}\right)^4=\boxed{0.51775}
\]
\item Having at least one double 1 in twenty-four rolls is the complement of not having any, so the probability is:
\[
P(\text{at least one double 1})=1-\left( \dfrac{35}{36} \right) ^4=\boxed{0.491404}
\]
\end{enumerate}

\end{solution}




%\begin{psexample}{Example Title}{}
%Example
%\end{psexample}
%
%\begin{pssolution*}{}{}
%Solution
%\end{pssolution*}
%
%\begin{psremark*}{Remark Title}{}
%Remark
%\end{psremark*}
%\begin{pproblem} \clock{0}{30}
%Problem statement
%\end{pproblem}

\end{document}