\documentclass[11pt]{article}
\usepackage[margin=1in]{geometry}
\usepackage[pdftex]{graphicx}
\usepackage{amsmath,amssymb,amsthm}
\usepackage{custom}

% color boxes
\usepackage[many]{tcolorbox}
\tcbuselibrary{theorems}
\newtcbtheorem{psexample}{Example}{colback=green!10,colframe=green!40!black,fonttitle=\bfseries,breakable,enhanced}{ex}
\newtcbtheorem{psproblem}{Problem}{colback=blue!10,colframe=blue!40!black,fonttitle=\bfseries,breakable,enhanced}{id}
\newtcbtheorem{psremark}{Remark}{colback=purple!10,colframe=purple!40!black,fonttitle=\bfseries,breakable,enhanced}{rm}
\newtcbtheorem{pssolution}{Solution}{colback=orange!10,colframe=orange!40!black,fonttitle=\bfseries,breakable,enhanced}{sn}
\tcbset{noparskip/.style={before={\par\pagebreak[0]\medskip\parindent=0pt},
after={\par\medskip}}}

% formatting for problems and solutions
\definecolor{ptred}{rgb}{0.7,0.1,0.1}
\newcommand{\ptfmt}[1]{\textbf{\color{ptred}#1\color{black}}}
\definecolor{advblue}{rgb}{0.1,0.1,0.7}
\newcommand{\advanced}{\!\textbf{\color{advblue}[A]\color{black}}\ }

\newtheorem*{solution}{Solution}
\newtheorem*{answer}{Answer}

\makeatletter
\newtheoremstyle{mystyle}
{\topsep}               % space above
{\topsep}               % space below
{}                      % bodyfont
{}                      % indent
{\bfseries}             % headfont
{}                      % punctuation
{0.6em}                 % space after head
{\llap{\hspace{.6em}}\thmname{#1}\thmnumber{ #2}\thmnote{\normalfont{ (#3)}}{\bfseries .}}  %theoremheadspec
\theoremstyle{mystyle}
\newtheorem{pproblem}{Problem}
\makeatother

% headers and footers
\usepackage{fancyhdr}
\pagestyle{fancy}
\lhead{Filip Ramazan}
\chead{}
\rhead{STAT 461}
\lfoot{}
\cfoot{\thepage}
\rfoot{}
\renewcommand{\headrulewidth}{0.4pt}
\setlength{\headheight}{14pt}

\newcommand{\psettitle}[1]{
    \begin{center}
    \huge \textbf{#1}
    \end{center}
}
\newcommand{\x}{\cdot}

\linespread{1.03} % give a little extra room
\setlength{\parindent}{0.2in} % reduce paragraph indent a bit

% uncomment to hide solutions
% \usepackage{environ}
% \NewEnviron{hide}{}
% \let\solution\hide
% \let\endsolution\endhide
% \let\answer\hide
% \let\endanswer\endhide

\begin{document}

\psettitle{Homework \#4}

\noindent

\begin{psproblem}{Dies by Calculation}{}
Suppose a fair die is tossed three times.
\begin{enumerate}[label=\alph*.]
\item Let $X$ be the largest of the faces that appear. Write with justification the probability density function of $X$.
\item Let $Y$ be the number of different faces that appear. Write with justification the probability density function
and the cumulative distribution function $F_Y$ of Y. Plot the graph of $F_Y$ .
\end{enumerate}
\end{psproblem}

\begin{solution}
\leavevmode
\begin{enumerate}[label=\alph*.]
\item Probability mass function of $X$: \[
\begin{array}{c|ccccccc}
    x & 0 & 1 & 2 & 3 & 4 & 5 & 6 \\ \hline
    \rule{0pt}{12pt} % Adds vertical space ONLY to this row without breaking the vertical line
    P(X = x) & \frac{1}{216} & \frac{7}{216} & \frac{19}{216} & \frac{37}{216} & \frac{61}{216} & \frac{91}{216} & \frac{127}{216}
\end{array}
\]
To find the probabilities for each value of $X$, we use the formula $\dfrac{k^3}{6^3}-\dfrac{(k-1)^3}{6^3}$, since $P(x \le k) - P(x \le k-1)=P(x=k)$
\item Probability mass function of $Y$: \[
\begin{array}{c|ccc}
    y & 1 & 2 & 3 \\ \hline
    \rule{0pt}{12pt} % Adds vertical space ONLY to this row without breaking the vertical line
    P(Y = y) & \frac{6}{216} & \frac{90}{216} & \frac{120}{216}
\end{array}
\]
\begin{flalign*}
P(x=1) &= 6 \x 1 \x 1 = 6 \text{, since there must be only 1 distinct number.} &&
P(x=2) &= 6 \x 5 \x 1 = 30 \text{, since there must be 2 distinct numbers, and one repeated.} &&
P(x=3) &= 6 \x 5 \x 4 = 120 \text{, since there must be 3 distinct numbers.} &&
\end{flalign*}
Cumulative distribution function of $Y$:
\[
F_{Y}(t)=
 \begin{cases} 
      0 & t < 1 \\
      \frac{1}{36} & 1 \le t < 2 \\
      \frac{4}{9} & 2 \leq t < 3 \\
      1 & 3 \le t 
   \end{cases}
\]



\end{enumerate}
\end{solution}




%\begin{psexample}{Example Title}{}
%Example
%\end{psexample}
%
%\begin{pssolution*}{}{}
%Solution
%\end{pssolution*}
%
%\begin{psremark*}{Remark Title}{}
%Remark
%\end{psremark*}
%\begin{pproblem} \clock{0}{30}
%Problem statement
%\end{pproblem}

\end{document}